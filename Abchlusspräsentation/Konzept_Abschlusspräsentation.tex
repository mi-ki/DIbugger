\documentclass[parskip=full]{scrartcl}
\setcounter{tocdepth}{2}
\usepackage[T1]{fontenc}    % avoid garbled Unicode text in pdf
\usepackage[utf8]{inputenc} % use utf8 file encoding for TeX sources
\usepackage[german]{babel}  % german hyphenation, quotes, etc
\usepackage{hyperref}       % detailed hyperlink/pdf configuration
\hypersetup{                % ‘texdoc hyperref‘ for options
pdftitle={PSE: Entwicklung eines relationalen Debuggers - Testbericht},%
,%
}
\usepackage{graphicx}       % provides commands for including figures
\usepackage{csquotes}       % provides \enquote{} macro for "quotes"
\usepackage[nonumberlist]{glossaries}     % provides glossary commands
\usepackage{enumitem}
\usepackage{pdfpages}
\usepackage{xcolor}
\newcommand\frage[1]{\textcolor{red}{#1}}
\renewcommand{\glstextformat}[1]{\textbf{\color{blue}\em #1}}
\font\myfont=cmr12 at 20pt

\title{
	\vspace{2cm}
	\myfont 
	Praxis der Softwareentwicklung:\\ 
	Entwicklung eines relationalen Debuggers\\
}
\subtitle{
	\vspace{1cm}
	\myfont
	Vorstellung des Produktes
}

\begin{document}
\clearpage
\maketitle
%\pagenumbering{gobble}
%\newpage

%\newpage
\pagenumbering{arabic}

\section{Einleitung}
Es wird in die Präsentation eingeleitet, indem dem Publikum die Aufgaben eines relationalen Debuggers vorgestellt werden.
\begin{itemize}
\item Erklärung der Anforderungen: Was war das Ziel? Was tut ein relationaler Debugger?
\item Einführen eines Beispiels, das während der Vorlesung als Programmcode wieder aufgegriffen wird
\item Herausforderungen, die sich aus der Aufgabenstellung ergaben (z.B. das Interpretieren und Ausführen von Code, Halten an den richtigen Stellen, etc.)
\end{itemize}
\section{Demonstration}
Das Produkt wird live vorgestellt. Das Beispiel aus der Einleitung wird wieder aufgegriffen.\\
Die folgenden Funktionalitäten werden vorgestellt:
\subsection{Allgemeine Einführung}
\begin{itemize}
\item Überblick über die Benutzeroberfläche
\item (Sprache ändern)
\item Erläutern der Mächtigkeit von Wlang mit Hilfe der Beispieltexte
\item Laden einer Konfigurationsdatei für die folgende Demonstration
\end{itemize}

\subsection{Übliche Debugger-Funktionen}
\begin{itemize}
\item Code einfügen
\item Breakpoints setzen
\item Debugging starten
\item Schritte machen
\end{itemize}

\subsection{Relationale Debugger-Funktionen}
\begin{itemize}
\item Watch-Expressions einfügen und auswerten
\item Bedingte Breakpoints einfügen und auswerten
\end{itemize}

\section{Kennzahlen}
Kennzahlen zum Produkt werden genannt.
\begin{itemize}
\item Klassen
\item Zeilen Code
\item Rekursionstiefe, Fehlerbehandlung, maximale Programmlänge
\item Anzahl an gleichzeitig debuggbaren Programmen
\item amüsante Kennzahlen
\end{itemize}

\section{Fazit}
Was das Team aus dem PSE mitgenommen hat.

\begin{itemize}
\item akkurate Aufwandschätzung ist schwierig
\item Synchronisation im Team ist auch mit git eine Herausforderung
\item Auch ein großes Projekt besteht aus Einzelschritten.
\item Die Frage \enquote{Was ist das Minimum Viable Product?} ist schwierig zu beantworten.
\item Wir haben im Ansatz erfahren, wie man einen Compiler baut.
\end{itemize}

\section{Schluss}
Das ganze Team kommt nach vorne.

\end{document}